\newcommand{\customDir}{}
\input{\customDir _LaTeX_master/LaTeX_master_setup.sty}

%\setboolean{twosided}{true}
%\setCustomDocumentClass{scrartcl}
%\setCustomDesign{htw}
%\setCustomSlidePath{Folien}

\setCustomTitle{Beleg UDP-Dateiübertragung}
\setCustomAuthor{Duy Tien Nguyen}
%\setCustomNoteA{TitlepageNoteBeforeAuthor}

\setCustomSignature{\textcolor{darkgray}{\customAuthor{}\\ s80287}}	% Formatierung der Signatur in der Fußzeile
\setCustomTitleAuthor{\customAuthor{} (s80287)}	% Formatierung des Autors auf dem Titelblatt

\input{\customDir _LaTeX_master/LaTeX_master.sty}
\input{\customDir _LaTeX_master/LaTeX_master_macros.sty}

%\bibliography{\customDir _Literatur/HTW_Literatur.bib}
%\setlength{\headheight}{10mm}	% default: ca. 8mm
\setlength{\footheight}{10mm}	% default: ca. 8mm


\usepackage{tikz-uml}

% TeXDoclet Compatibility:
\makeatletter
\DeclareOldFontCommand{\rm}{\normalfont\rmfamily}{\mathrm}
\DeclareOldFontCommand{\sf}{\normalfont\sffamily}{\mathsf}
\DeclareOldFontCommand{\tt}{\normalfont\ttfamily}{\mathtt}
\DeclareOldFontCommand{\bf}{\normalfont\bfseries}{\mathbf}
\DeclareOldFontCommand{\it}{\normalfont\itshape}{\mathit}
\DeclareOldFontCommand{\sl}{\normalfont\slshape}{\@nomath\sl}
\DeclareOldFontCommand{\sc}{\normalfont\scshape}{\@nomath\sc}
\makeatother

% TeXDoclet Preamble:
%\input{javadoc/TeXDoclet_preamble.tex}
\usepackage{graphics}
\usepackage{ifpdf}
\lstset{language=Java,breaklines=true}

\newcommand{\entityintro}[3]{%
  \hbox to \hsize{%
    \vbox{%
      \hbox to .2in{}%
    }%
    {\bf  #1}%
    \dotfill\pageref{#2}%
  }
  \makebox[\hsize]{%
    \parbox{.4in}{}%
    \parbox[l]{5in}{%
      \vspace{1mm}%
      #3%
      \vspace{1mm}%
    }%
  }%
}
\newcommand{\refdefined}[1]{
\expandafter\ifx\csname r@#1\endcsname\relax
\relax\else
{$($in \ref{#1}, page \pageref{#1}$)$}\fi}
\chardef\textbackslash=`\\

\begin{document}

%\selectlanguage{english}
\maketitle
\newpage
\tableofcontents
\newpage

\chapter{Dokumentation}

\section{Programmierung}
\label{programing}

\subsection{Anforderungen}
Die Programmierung wurde nach den Anforderungen der Aufgabenstellung gelöst. Diese lauten wie folgt:
\begin{itemize}
\item Client:
\begin{itemize}
\item Über Konsole mit Parametern „Zieladresse“, „Portnummer“ und „Dateiname“ aufrufbar (zum debuggen auch „Paketverlust“ und „mittlere Verzögerung“)
\item Zeigt während und nach der Übertragung jede Sekunde die Datenrate an
\item Korrigiert Fehler bei verlorenen oder vertauschten Paketen:\\
verlorenes ACK: Datenpaket erneut senden, vertauschtes ACK: vorhergehendes Datenpaket senden.
\end{itemize}
\item Server:
\begin{itemize}
\item Über Konsole mit dem Parameter „Portnummer“ aufrufbar (zum debuggen auch „Paketverlust“ und „mittlere Verzögerung“)
\item Speichert Datei in seinem Pfad unter dem korrekten Dateiname (Zeichen „1“ wird angehängt, wenn Datei bereits existiert)
\item Korrigiert Fehler bei verlorenen oder vertauschten Paketen:\\
verlorenes Datenpaket: kein ACK senden, vertauschtes Datenpaket: ACK des vorhergehenden Datenpakets senden.
\end{itemize}
\end{itemize}

%Abgrenzung
\subsection{Probleme/Limitierungen/Verbesserungsvorschläge}
\label{abgrenz}
Unten sind die Probleme/Limitierungen/Verbesserungsvorschläge für das Protokoll.
\begin{itemize}
\item Der Server kann nicht mehrere Clients auf einmal bedienen.
\item Der Client kann nicht wissen, ob die Datei tatsächlich fertig übertragen wurde, wenn das letzte ACK-Paket vom Server verloren geht.
\item Die Klasse FileCopy wird hier nicht hinzugefügt.
\end{itemize}

\section{Zustandsdiagramme}
\label{stated}
\subsection{Client}
\begin{center}
\begin{tikzpicture}
\umlstateinitial[y=3,name=start]
\umlbasicstate[y=0,name=startupload, fill=white]{Initialisiere Upload}
\umltrans{start}{startupload}
\umlstatefinal[x=6,y=-2,name=finalerror]
\umlHVtrans[arg={Fehler},pos=.5, anchor2=90]{startupload}{finalerror}
%\umlbasicstate[y=-4,name=sendfirst, fill=white]{Sende erstes Paket}
%\umltrans[arg={Ok},left, pos=.5]{startupload}{sendfirst}
%\umltrans[arg={Timeout},pos=.5]{sendfirst}{finalerror}
\umlbasicstate[y=-4,name=sendpacket, fill=white]{Sende Paket}
\umltrans[arg={Ok},left, pos=.5]{startupload}{sendpacket}
\umlHVtrans[arg={Timeout mit max. Anzahl Versuche},pos=.1, below right, anchor2=-90]{sendpacket}{finalerror}
\umltrans[recursive=160|200|2.5cm, recursive direction = right to right, left, arg={ACK empfangen || Timeout},pos=1.5]{sendpacket}{sendpacket}
\umlstatefinal[y=-7,name=finalok]
\umltrans[arg={ACK zu letztem Paket empfangen}, pos=.5, left]{sendpacket}{finalok}
\end{tikzpicture}
\end{center}

\subsection{Server}
\begin{center}
\begin{tikzpicture}
\umlstateinitial[y=3,name=start]
\umlbasicstate[y=0,name=waitpacket, fill=white]{Warte auf Paket}
\umltrans{start}{startupload}
%\umlbasicstate[y=-4,name=sendfirst, fill=white]{Sende erstes Paket}
%\umltrans[arg={Ok},left, pos=.5]{startupload}{sendfirst}
%\umltrans[arg={Timeout},pos=.5]{sendfirst}{finalerror}
\umlbasicstate[y=-4,name=processpacket, fill=white]{Verarbeite Paket}
\umltrans[arg={Paket empfangen},left, anchor1=-110, anchor2=110, pos=.5]{startupload}{processpacket}
\umlbasicstate[y=-8,name=sendack, fill=white]{Sende ACK}
\umltrans[arg={Ok || Falsches Paket}, left, pos=.5]{processpacket}{sendack}
\umltrans[arg={Fehler},right, anchor1=70, anchor2=-70, pos=.5]{processpacket}{waitpacket}
\umlHVHtrans[right, arg={},pos=1.5,anchor1=180, anchor2=180, arm1=-2.5cm]{sendack}{startupload}
\umlbasicstate[y=-12,name=processfile, fill=white]{Verarbeite Datei}
\umltrans[arg={Paket hat Datei-CRC}, left, pos=.5, anchor1=-110, anchor2=110]{sendack}{processfile}
\umlHVHtrans[right, arg={Fehler},pos=1.5,anchor1=0, anchor2=0, arm1=1.5cm]{processfile}{startupload}
\umltrans[arg={Ok},right, anchor1=70, anchor2=-70, pos=.5]{processfile}{sendack}
\end{tikzpicture}
\end{center}

%\chapter{JavaDoc}
%\selectlanguage{english}
%\label{javadoc}
%\input{javadoc/TeXDoclet.tex}

%\newpage
%\printbibliography

\end{document}